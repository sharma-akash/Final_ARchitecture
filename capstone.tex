\documentclass[onecolumn, draftclsnofoot,10pt, compsoc]{IEEEtran}
\usepackage{graphicx}
\usepackage{url}
\usepackage{setspace}

\usepackage{geometry}
\geometry{textheight=9.5in, textwidth=7in}

% 1. Fill in these details
\def \CapstoneTeamName{			Augmented Reality Project}
\def \CapstoneTeamNumber{		60}
\def \GroupMemberOne{			Akash Sharma}
\def \GroupMemberTwo{			Ross Shoger}
\def \GroupMemberThree{			Sean Wilton}
\def \CapstoneProjectName{		An Augmented reality framework  for Sensor Data Visualizations }
\def \CapstoneSponsorCompany{	Oregon State University EECS}
\def \CapstoneSponsorPerson{	Raffaele De Amicis}

% 2. Uncomment the appropriate line below so that the document type works
\def \DocType{		Problem Statement
				%Requirements Document
				%Technology Review
				%Design Document
				%Progress Report
				}
			
\newcommand{\NameSigPair}[1]{\par
\makebox[2.75in][r]{#1} \hfil 	\makebox[3.25in]{\makebox[2.25in]{\hrulefill} \hfill		\makebox[.75in]{\hrulefill}}
\par\vspace{-12pt} \textit{\tiny\noindent
\makebox[2.75in]{} \hfil		\makebox[3.25in]{\makebox[2.25in][r]{Signature} \hfill	\makebox[.75in][r]{Date}}}}
% 3. If the document is not to be signed, uncomment the RENEWcommand below
\renewcommand{\NameSigPair}[1]{#1}

%%%%%%%%%%%%%%%%%%%%%%%%%%%%%%%%%%%%%%%
\begin{document}
\begin{titlepage}
    \pagenumbering{gobble}
    \begin{singlespace}
    	%\includegraphics[height=4cm]{coe_v_spot1}
        \hfill 
        % 4. If you have a logo, use this includegraphics command to put it on the coversheet.
        %\includegraphics[height=4cm]{CompanyLogo}   
        \par\vspace{.2in}
        \centering
        \scshape{
            \huge CS Capstone \DocType \par
            {\large\today}\par
			{\large CS461 Fall 2017}\par
            \vspace{.5in}
            \textbf{\Huge\CapstoneProjectName}\par
            \vfill
            {\large Prepared for}\par
            \Huge \CapstoneSponsorCompany\par
            \vspace{5pt}
            {\Large\NameSigPair{\CapstoneSponsorPerson}\par}
            {\large Prepared by }\par
            Group\CapstoneTeamNumber\par
            % 5. comment out the line below this one if you do not wish to name your team
            \CapstoneTeamName\par 
            \vspace{5pt}
            {\Large
                \NameSigPair{\GroupMemberOne}\par
                %\NameSigPair{\GroupMemberTwo}\par
                %\NameSigPair{\GroupMemberThree}\par
            }
            \vspace{20pt}
        }
        \begin{abstract}
        % 6. Fill in your abstract    
        To preface, our team has not yet met with the client, and thus the following text shall primarily discuss the given project description. Our mentor, Raffaele De Amicis, is affiliated with EECS here at Oregon State and our project deals with augmented reality. This text discusses the motivation behind the project, and what issue is trying to be solved. In short, we are maximizing building efficiency along with other infrastructures. On top of that, we go into detail about how augmented reality will be utilized and investigated in order to produce useful data collection and analysis. Lastly, we discuss some of the requirements we need to meet to satisfy the client.     
        \end{abstract}     
    \end{singlespace}
\end{titlepage}
\newpage
\pagenumbering{arabic}
\tableofcontents
% 7. uncomment this (if applicable). Consider adding a page break.
%\listoffigures
%\listoftables
\clearpage

% 8. now you write!
\section{Definition and Description of Problem}
Ultimately, it appears the desired outcome is to create building models to ensure new buildings that are cost-effective and reliable. In a sense, we are trying to maximize the efficiency in cities where a majority of the population resides and where a lot of the CO2 emissions occur. We shall use the immersive technology of AR to see what we can do in terms of improving data and models. Essentially it is a project that links digital information with physical characteristics. 
	This project works with data in buildings within urban areas and this data is collected with sensors that allow analysis of climate, structure, and more. Through data, we can create more advanced models than the current building simulation tools. We can have the user be placed within the data and analyze from there rather than making inferences and separating the two. It proposes to implement a full-scale real-time Sensor Data Visualizations Framework, with the purpose of contributing to the body of knowledge on the structural and durability performance of infrastructure. Our platform will be an AR application where the visitor to the building can have access to all the sensor data related to elements in a building which in turn tells us a great deal about the structure’s performance. 

	

\section{Proposed Solution}
Our goal is to develop the management services platform with ensures data collection and ingestion. In other words, we will be developing the middle-man between the data itself and the processing. Further, we shall implement an augmented visualization that can be used to analyze said data. 
	For performance metrics, we know that we must have 5 key points. Data exchange System architecture, Database modelling and implementation, Data integration platform modelling and implementation, Human Behavior data collection tool, Virtual and Augmented Reality Structure Monitoring Visualizations software

	
\end{document}





